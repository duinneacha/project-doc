\chapter{Discussion and Conclusion}
\label{chap:conclusion}
\lhead{\emph{Discussion and Conclusion}}

\section{Introduction}

Briefly reintroduce the research's objectives and the methodologies employed.





\section{Discussion of Findings}
\subsection{OCR Performance}

Discuss the overall performance of the Tesseract OCR system on the raw image datasets.
Highlight the improvements observed after implementing the various pre-processing techniques.
Contrast the results between different image folders, noting any significant variations.

\subsection{Image Pre-processing Techniques}

Delve into the significance of each pre-processing technique:



Colour masks (Red and Green masks)
Grayscale conversion
Resizing
Thresholding techniques
Denoising
Deblurring
Others
Discuss the specific impact of each technique on OCR performance and any trade-offs observed.

\subsection{CRNN Model Analysis}

Discuss the benefits and challenges of using the CRNN model.
Contrast its performance with the Tesseract OCR system.

\section{Challenges and Limitations}

Detail any challenges faced during the research, e.g., images with diverse properties, environmental conditions affecting image quality, etc.
Discuss limitations in the methodologies used and potential areas for improvement.


\section{Implications and Applications}

Discuss the broader implications of your findings. How do they contribute to the field of OCR or specific applications like reading sensor data?
Mention potential real-world applications and benefits.
\section{Conclusions}

Summarize the main findings of your research.
Reiterate the significance of your research in enhancing the performance of OCR systems.


\section{Recommendations and Future Work}

Provide recommendations based on your findings, especially for researchers or practitioners aiming to utilize OCR for similar applications.
Discuss potential areas of future research or further enhancements that could be explored.




