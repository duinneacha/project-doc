\chapter{Introduction}
\label{chap:intro}
\lhead{\emph{Introduction}}


\section{Area of Interest}


The area of interest for this literature review is the intersection of computer vision, optical character recognition (OCR), and deep learning, with particular emphasis on the Tesseract OCR engine and Convolutional Recurrent Neural Networks (CRNNs). These technological advancements have revolutionized the way machines recognize and understand visual information, especially digits. Given their diverse and significant applications, ranging from digitizing written documents to aiding autonomous vehicle navigation, they hold vast potential for transforming many sectors. This research focuses on exploring the principles that underlie these tools, their performance in real-world applications, and the possibilities they offer for future development. This involves assessing the strengths of these systems, identifying their limitations, and suggesting potential areas of improvement. Moreover, it considers how these technologies are pushing the boundaries of OCR, paving the way for more sophisticated and versatile tools that can better navigate the complexities and variations in text size, font, and orientation often encountered in different visual scenes.

\newcommand{\startpicsWH}[1]{\includegraphics[width=0.15\textwidth,height=0.1\textheight]{#1}}

\begin{table}[ht]
    \centering
    \begin{tabular}{ccc}
        \startpicsWH{Figures/start_pics/IMG-20200219-WA0002.jpg} & \startpicsWH{Figures/start_pics/IMG-20200220-WA0000.jpg} & \startpicsWH{Figures/start_pics/IMG-20200220-WA0002.jpg} \\
        \startpicsWH{Figures/start_pics/IMG-20200220-WA0003.jpg} & \startpicsWH{Figures/start_pics/IMG-20200220-WA0004.jpg} & \startpicsWH{Figures/start_pics/IMG-20200220-WA0005.jpg} \\
        \startpicsWH{Figures/start_pics/IMG-20200220-WA0006.jpg} & \startpicsWH{Figures/start_pics/IMG-20200220-WA0007.jpg} & \startpicsWH{Figures/start_pics/IMG-20200220-WA0009.jpg} \\
        \startpicsWH{Figures/start_pics/IMG-20200220-WA0010.jpg} & \startpicsWH{Figures/start_pics/IMG-20200220-WA0011.jpg} & \startpicsWH{Figures/start_pics/IMG-20200220-WA0013.jpg} \\
    \end{tabular}
    \caption{Nimbus Sensor Images}
    \label{table:image-table}
\end{table}


Optical Character Recognition (OCR) technology has seen substantial advancements in recent years, transforming the process of data extraction from visual mediums to digital formats. This technology, crucial in numerous fields ranging from document digitization to automated data entry systems. OCR holds specific importance when it comes to interpreting sensor readings, a key aspect of data-driven industries. The necessity for accurate, efficient, and automated reading of sensor-generated data has led to the investigation of various techniques and models within the OCR domain.

Two models which feature prominently emerged as potential solutions, namely Tesseract, an open-source OCR engine sponsored by Google, and Convolutional Recurrent Neural Network (CRNN), a combination of CNN, RNN, and Connectionist Temporal Classification that offers promising results in scene text recognition tasks.

In OCR applications, image preprocessing has a pivotal role. It prepares an image for further processing by reducing noise and unnecessary details and enhancing features that are important for later stages, thereby directly influencing the accuracy of the final output. Among various preprocessing techniques, the novel approach of red and green color masking, followed by conversion to grayscale, has shown to significantly improve the accuracy of digit recognition.




\section{Motivation}


The motivation behind this research stems from the challenges encountered in the manual and infrequent readings of environmental sensors in various operational settings such as factories. These sensors, while accurate and essential, lack a means for continuous data capture. Typically, an individual manually reads the sensor outputs at fixed intervals, which could range from hourly to daily. This method, while necessary, is prone to human error, potentially leading to inaccuracies in the recorded data and subsequent analysis reports. Furthermore, the infrequency of readings may result in delays in responding to critical sensor data, which could precipitate further issues. These complications could be mitigated with the implementation of Optical Character Recognition (OCR) technology. By enabling continuous, automated readings of these sensors, OCR has the potential to not only reduce errors but also ensure timely reaction to important sensor changes, optimizing the overall operation and efficiency of the systems.

\section{Aims and Objectives}
\section{Structure of the Thesis}

% Putting in comments within the TeX file can be really useful in making notes for yourself and dumping text that you intend to edit later
