\chapter{Introduction}
\label{chap:intro}
\lhead{\emph{Introduction}}


Optical Character Recognition (OCR) technology has seen substantial advancements in recent years, transforming the process of data extraction from visual mediums to digital formats. This technology, crucial in numerous fields ranging from document digitization to automated data entry systems, holds specific importance when it comes to interpreting sensor readings, a key aspect of data-driven industries. The necessity for accurate, efficient, and automated reading of sensor-generated data has led to the investigation of various techniques and models within the OCR domain.

Two models have prominently emerged as potential solutions, namely Tesseract, an open-source OCR engine sponsored by Google, and Convolutional Recurrent Neural Network (CRNN), a combination of CNN, RNN, and Connectionist Temporal Classification that offers promising results in scene text recognition tasks.

In OCR applications, image preprocessing has a pivotal role. It prepares an image for further processing by reducing noise and unnecessary details and enhancing features that are important for later stages, thereby directly influencing the accuracy of the final output. Among various preprocessing techniques, the novel approach of red and green color masking, followed by conversion to grayscale, has shown to significantly improve the accuracy of digit recognition.

In addition to these techniques, the selection of the correct font for each sensor is another critical element that affects the accuracy of the OCR system. Despite its importance, this aspect has been less emphasized in existing literature, thereby forming a crucial area of exploration in this study.

This literature review explores the current state of OCR technologies, with a particular focus on Tesseract and CRNN models. It delves into various image preprocessing techniques, emphasizing the unique method of red and green color masking before conversion to grayscale. Lastly, it investigates the role of font selection in enhancing OCR accuracy, thereby setting the context for the subsequent research.

% Putting in comments within the TeX file can be really useful in making notes for yourself and dumping text that you intend to edit later

\section{Motivation}
Why is it important to do a project on this topic? This should cover your key motivation for this. For example an excellent student from 2016 noticed a large number of homeless sleeping rough in Cork and was motivated to develop a system that load balanced the homeless shelters to try to accommodate the maximum number of homeless. This section can include the personal pronoun but the rest of the report should be third person passive, this is the case with most technical reports! For example here it is fine to say "... I decided to develop and app to help ...".

\section{Contribution}
Enumerate the main contributions. Here try to zoom out, to talk from the perspective of a Computer Science graduate. In other words, imagine you are talking to a job panel, and you want to show your computer science skills by enumerating how they are reflected in your project work. A good guide here is to look back over the modules you have covered as an undergrad from 2/3rd year, how many tools and techniques from these modules do you have in the project and to what extent? How have you advanced beyond the module content? Do you have anything new?

\section{Structure of This Document}
% notice how I cross referenced the chapters through using the \label tag --> LaTeX is VERY similar to HTML and other mark up languages so you should see nothing new here!
This section is quite formulaic. Briefly describe the structure of this document, enumerating what does each chapter and section stands for. For instance in this work in Chapter \ref{chap:background} the guidance in structuring the literature review is given. Chapter \ref{chap:problem} describes the main requirements for the problem definition and so on ...