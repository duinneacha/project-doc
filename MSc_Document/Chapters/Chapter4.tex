\chapter{Results}
\label{chap:results}
\lhead{\emph{Results}}

\section{Introduction}

This chapter delves into the empirical findings derived from the systematic approach delineated earlier. Using the raw image datasets subjected to various pre-processing techniques, the performance of the OCR systems will be analysed and presented. The results aim to validate the hypothesis that image pre-processing can significantly enhance the efficacy of OCR systems on sensor reading images. Each section will provide a detailed account of the outcomes, offering insights into the effectiveness of the applied methodologies and setting the stage for the subsequent discussion and analysis.

\section{First Sprint - Global Generic}
\section{Second Sprint - Global Generic Analysis Resized}
\section{Analysis Tesseract Separate Folders}

\begin{table}[h]
    \centering
    \caption{OCR Performance for Different Folders}
    \label{tab:ocr_performance}
    \begin{tabular}{|l|c|c|c|c|c|c|}
        \hline
        \textbf{Folder} & \textbf{Total Count} & \multicolumn{2}{c|}{\textbf{Tesseract}} & \multicolumn{3}{c|}{\textbf{CRNN}}                                                            \\
        \hline
                        &                      & \textbf{Read}                           & \textbf{Not Read}                  & \textbf{Read} & \textbf{Not Read} & \textbf{No Contours} \\
        \hline
        A               & 160                  & 70                                      & 90                                 & 46            & 9                 & 105                  \\
        B               & -                    & -                                       & -                                  & -             & -                 & -                    \\
        C               & -                    & -                                       & -                                  & -             & -                 & -                    \\
        \hline
    \end{tabular}
\end{table}


\newpage

\subsection{Sipa 2}
\subsection{Sipa 3}
\subsection{Sipa 4}
\subsection{Sipa 5}
\subsection{Sipa 6}
\subsection{Sipa 7}
\subsection{Sipa 8}
\subsection{Sipa 9}
\subsection{Sipa 10}
\subsection{Sipa 11}

